\section{Marco Teórico}
El propósito de este marco teórico es introducir los conceptos clave que permitan abordar
la problemática de la detección de incendios forestales a partir del análisis de imágenes
con técnicas de aprendizaje profundo. Se explorarán las capacidades de las redes neuronales,
específicamente las redes neuronales convolucionales (CNNs), como herramientas a usar
para la detección y mitigación de estos eventos.

\subsection{Redes Neuronales Convolucionales}
Las redes neuronales convolucionales (CNNs) son un tipo de arquitectura de aprendizaje
profundo altamente efectiva para el procesamiento de datos que se pueden representar
como una cuadrícula, como lo son las imágenes. Su diseño les permite aprender de forma
automática jerarquías espaciales y patrones relevantes, lo cual las convierte en una
herramienta ideal para la claificación de imágenes y la detección de objetos.

Su arquitectura generalmente consta de tres tipos de capas principales: convolucionales,
de agrupación (\textit{pooling}) y completamente conectadas. Las capas convolucionales
aplican filtros a la imagen de entrada para extraer características relevantes, mientras
que las de agrupación reducen la dimensionalidad de las características extraídas.
Finalmente, las capas completamente conectadas se encargan de la clasificación final,
al asegurar que cada neurona de salida esté conectada a todas las neuronas de la capa
anterior.

El uso de CNNs en la detección de humo de incendios forestales se ha vuelto clave gracias
a su capacidad para procesar grandes volúmenes de datos visuales, como imágenes sateliteles
y transmisiones de cámaras de vigilancia. Esto permite diferenciar el humo de elementos
como nubes o niebla, y de forma más eficiente a diferencia de métodos tradicionales.

\subsection{YOLO}
You Only Look Once es un modelo avanzaado de detección de objetos diseñado para identificar
y localizar elemntos en imágenes en tiempo real con alta precisión. Su funcionamiento se
basa en redes neuronales convolucionales (CNNs) para analizar imágenes con un solo
procesamiento, dividiéndolas en cuadrículas y prediciendo la posición, dimensiones y
clase de los objetos detectados.

YOLOv8, la versión más reciente, introduce un enfoque sin anclas (\textit{anchor-free}),
eliminando las posiciones predefinidas de las cajas delimitadoras usadas en versiones
anteriores. Esto le ayuda a simplificar el entrenamiento y mejorar la precisión en la
detección de objetos con formas y tamaños variables. Además, incorpora CSPNet
(\textit{Cross-Stage Partial Networks}) como backbone, una arquitectura que optimiza el
flujo de información y reutiliza características de capas anteriores para reducir las
redundancias y mejorar la eficiencia computacional.

Estas características le permiten a YOLOv8 servir como herramienta altamente eficaz para
tareas en tiempo real, y para actividades de vigilancia y monitoreo ambiental como es el
caso de la detección de incendios forestales.

\subsection{Transfer Learning}
El aprendizaje por transferencia es una técnica de aprendizaje automático que consiste
en reutilizar conocimientos aprendidos en un dominio para mejorar el rendimiento en otro
dominio relacionado. En el contexto de las redes neuronales, esto implica tomar una red
preentrenada en un conjunto de datos grande y ajustarla a un conjunto de datos más
pequeño y específico.
